\usepackage{booktabs}
\usepackage[utf8]{inputenc}
%% \usepackage[spanish]{babel}  cambio este por el de abajo que soluciona el problema de las comillas
\usepackage[spanish,shorthands=off]{babel} 
\usepackage{amsthm} %% math
\usepackage{fvextra}
\usepackage{rotating}
\DefineVerbatimEnvironment{Highlighting}{Verbatim}{breaklines,commandchars=\\\{\}, fontsize=\small}







% %%  ------------------  Tabla 
% Tabla en lugar de cuadro
\gappto\captionsspanish{\renewcommand{\tablename}{Tabla}
          \renewcommand{\listtablename}{Índice de tablas}}



%% -------------------- tamaño ------------- inicialmente 170 x 240

%\usepackage[paperwidth = 170mm,
%			paperheight = 240mm, top=3.25cm, bottom=2.5cm, left=2.5cm, right=2cm]{geometry}

\usepackage[paperwidth = 190mm,
			paperheight = 250mm, top=3cm, bottom=2.5cm, left=2.5cm, right=2cm]{geometry}
\usepackage[a4, center, cam]{crop}
\makeatother


% headers: centra el encabezado y en minúsucula
%  -------------------------------------------------------
\usepackage{fancyhdr}

\pagestyle{fancy}
\fancyhf{}
\fancyhead[CO]{\nouppercase{\emph{\rightmark}}}
\fancyhead[CE]{\nouppercase{\emph{\leftmark}}}
\fancyhead[RO]{\thepage}
\fancyhead[LE]{\thepage}
% no line in header or footer
\renewcommand{\headrulewidth}{0pt}


% Code chunk mods -------------------------------------------------------------

%% color texto de comentarios
%\renewcommand{\CommentTok}[1]{\textcolor[rgb]{0.38,0.63,0.69}{\textit{{#1}}}}
% está chulo pero el que mejor queda es el estandar, a no ser que cambiase el fondo de la chunk
%\renewcommand{\CommentTok}[1]{\textcolor[rgb]{1, 0.7, 0.4}{\textbf{#1}}}



% ----------- color encabezado 
\usepackage{xcolor}
\usepackage{sectsty}



%\definecolor{colchap}{rgb}{1, 0.7, 0.4}
\definecolor{colchap}{HTML}{A16C00}
\chapterfont{\color{colchap}}  % sets colour of chapters         
\sectionfont{\color{cyan}}  % sets colour of sections
\subsectionfont{\color{teal}}  % sets colour of sections 
\subsubsectionfont{\color{darkgray}}  % sets colour of sections

%\definecolor{colsect}{rgb}{0.86, 0.65, 0.12}
%\sectionfont{\color{colsect}}  % sets colour of sections



%% ---------------- inline code -----------
\usepackage{xcolor}
%% \definecolor{bgcolor}{HTML}{ffcb7c} descomenta para poner el color de subrayado
\definecolor{tcolor}{HTML}{A16C00}
\let\oldtexttt\texttt
\renewcommand{\texttt}[1]{\textcolor{tcolor}{{\oldtexttt{#1}}}}


%% ---------------- blocks -----------

\usepackage{tcolorbox}

% Redefinir estilo para callout de tipo "note"
\definecolor{infoboxpink}{RGB}{255,214,235} % mismo tono que en HTML
\definecolor{infoboxcyan}{RGB}{0,255,255}

% \tcbset{
%   quarto.note/.style={
%     colback=infoboxpink,
%     colframe=infoboxcyan,
%     coltext=black,
%     boxsep=5pt,
%     arc=4pt,
%     left=6pt,
%     right=6pt,
%     top=6pt,
%     bottom=6pt,
%   }
% }
% 
% % Definir entorno que Quarto usará para los callouts tipo "note"
% \newtcolorbox{calloutnote}[1][]{quarto.note, #1}


% 
% \newtcolorbox{infobox}{
%   colback= pink!45!white,
%   colframe= cyan,
%   coltext=black,
%   boxsep=5pt,
%   arc=4pt}
%   
% 
% \newtcolorbox{infobox_blue}{
%   colback= blue!5!white,
%   colframe= blue!75!black,
%   coltext=black,
%   title=Nota,
%   boxsep=5pt,
%   arc=4pt}
%     
% 
% \newtcolorbox{infobox_resume} {
%  colback=yellow!40!white,
%   colframe=teal,
%   coltext=black,
%   boxsep=5pt,
%   arc=4pt}
%   
  %%%%  -------------  color de texto -----------
  
\usepackage{xcolor}

\newcommand{\textred}[1]{\textcolor{red}{#1}}

\newcommand{\textblue}[1]{\textcolor{blue}{#1}}

  
  
  %% ---------------- índice de contenidos -----------

 \usepackage{makeidx}
 \makeindex

  